\documentclass[landscape]{article}
\usepackage{lmodern}
\usepackage{amssymb,amsmath}
\usepackage{ifxetex,ifluatex}
\usepackage{fixltx2e} % provides \textsubscript
\ifnum 0\ifxetex 1\fi\ifluatex 1\fi=0 % if pdftex
  \usepackage[T1]{fontenc}
  \usepackage[utf8]{inputenc}
\else % if luatex or xelatex
  \ifxetex
    \usepackage{mathspec}
  \else
    \usepackage{fontspec}
  \fi
  \defaultfontfeatures{Ligatures=TeX,Scale=MatchLowercase}
\fi
% use upquote if available, for straight quotes in verbatim environments
\IfFileExists{upquote.sty}{\usepackage{upquote}}{}
% use microtype if available
\IfFileExists{microtype.sty}{%
\usepackage{microtype}
\UseMicrotypeSet[protrusion]{basicmath} % disable protrusion for tt fonts
}{}
\usepackage[margin=1cm]{geometry}
\usepackage{hyperref}
\hypersetup{unicode=true,
            pdftitle={Commandes usuelles pour calcul intégral en MAXIMA},
            pdfborder={0 0 0},
            breaklinks=true}
\urlstyle{same}  % don't use monospace font for urls
\usepackage{graphicx,grffile}
\makeatletter
\def\maxwidth{\ifdim\Gin@nat@width>\linewidth\linewidth\else\Gin@nat@width\fi}
\def\maxheight{\ifdim\Gin@nat@height>\textheight\textheight\else\Gin@nat@height\fi}
\makeatother
% Scale images if necessary, so that they will not overflow the page
% margins by default, and it is still possible to overwrite the defaults
% using explicit options in \includegraphics[width, height, ...]{}
\setkeys{Gin}{width=\maxwidth,height=\maxheight,keepaspectratio}
\IfFileExists{parskip.sty}{%
\usepackage{parskip}
}{% else
\setlength{\parindent}{0pt}
\setlength{\parskip}{6pt plus 2pt minus 1pt}
}
\setlength{\emergencystretch}{3em}  % prevent overfull lines
\providecommand{\tightlist}{%
  \setlength{\itemsep}{0pt}\setlength{\parskip}{0pt}}
\setcounter{secnumdepth}{0}
% Redefines (sub)paragraphs to behave more like sections
\ifx\paragraph\undefined\else
\let\oldparagraph\paragraph
\renewcommand{\paragraph}[1]{\oldparagraph{#1}\mbox{}}
\fi
\ifx\subparagraph\undefined\else
\let\oldsubparagraph\subparagraph
\renewcommand{\subparagraph}[1]{\oldsubparagraph{#1}\mbox{}}
\fi

%%% Use protect on footnotes to avoid problems with footnotes in titles
\let\rmarkdownfootnote\footnote%
\def\footnote{\protect\rmarkdownfootnote}

%%% Change title format to be more compact
\usepackage{titling}

% Create subtitle command for use in maketitle
\newcommand{\subtitle}[1]{
  \posttitle{
    \begin{center}\large#1\end{center}
    }
}

\setlength{\droptitle}{-2em}
  \title{Commandes usuelles pour calcul intégral en \texttt{MAXIMA}}
  \pretitle{\vspace{\droptitle}\centering\huge}
  \posttitle{\par}
  \author{}
  \preauthor{}\postauthor{}
  \date{}
  \predate{}\postdate{}

\usepackage{longtable}
\usepackage{booktabs}
\usepackage{multirow}

\begin{document}
\maketitle

\begin{center}
\begin{longtable}{@{}p{4cm}p{12cm}cc@{}}
\toprule
\textbf{Sujet} & \textbf{Discussion} & \textbf{Entrée \texttt{MAXIMA}} & \textbf{Sortie \texttt{MAXIMA}} \\ 
\midrule
\endhead

\textbf{Entrée de commandes} & Vous devez utiliser le point virgule (\texttt{;}) et \texttt{Ctrl-Entrée} pour effectuer une commande &
5/6*5/6; & $\frac{25}{36}$ \\

\midrule

\textbf{Utilisation de la ligne précédente} & Le symbole (\texttt{\%}) dit à \texttt{MAXIMA} d'utiliser le calcul précédent. &
\%+1; & $\frac{61}{36}$ \\

\midrule

\textbf{Utilisation d'une ligne  par son nom} & Vous pouvez utiliser un résultat à l'aide de sa ligne de sortie. &
\%o2+1; & $\frac{97}{36}$ \\

\midrule

\textbf{Évaluation numérique} & Vous pouvez demander à \texttt{MAXIMA} de calculer la valeur numérique d'un résultat. &
\texttt{float(\%)}; & $2.694444444444445$ \\

\midrule

\multirow{5}{*}{\textbf{Évaluation numérique}} & Addition : \texttt{+}. & \texttt{1+3}; & $4$ \\
& Soustraction : \texttt{-}. & \texttt{5-10}; & $-5$ \\
& Multiplication : \texttt{+}. & \texttt{2*3}; & $6$ \\
& Division : \texttt{/}. & \texttt{5/25}; & $\frac{1}{5}$ \\
& Puissance : \^. & \texttt{3\^2}; & $9$ \\

\midrule

\textbf{Définir une fonction} & Pour définir une fonction, vous lui donnez un nom, suivi par sa variable indépendante entre parenthèses, suivis par le symbole \texttt{:=}, suivi par sa définition. &
\texttt{f1(x):=x\^2-5*x+6}; & $f1(x):=x^2-5x+6$ \\

\midrule

\textbf{Évaluation d'une fonction} & Une fois définie, vous pouvez utiliser la fonction de la même façon que vous le feriez habituellement. &
\texttt{f1(5)}; & $6$ \\

\midrule

\multirow{2}{*}{\textbf{Assignation d'une valeur à une variable}} & Le symbole \texttt{:} assigne une valeur à une variable. & \texttt{a:5}; & $5$ \\
& La valeur associée à la variable sera maintenant utilisée à la place du nom de la variable. & \texttt{f1(a)}; & $6$ \\

\bottomrule
\end{longtable}
\end{center}


\end{document}
