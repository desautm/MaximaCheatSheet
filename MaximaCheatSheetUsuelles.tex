\documentclass[8pt,landscape]{article}
\usepackage{lmodern}
\usepackage{amssymb,amsmath}
\usepackage{ifxetex,ifluatex}
\usepackage{fixltx2e} % provides \textsubscript
\ifnum 0\ifxetex 1\fi\ifluatex 1\fi=0 % if pdftex
  \usepackage[T1]{fontenc}
  \usepackage[utf8]{inputenc}
\else % if luatex or xelatex
  \ifxetex
    \usepackage{mathspec}
  \else
    \usepackage{fontspec}
  \fi
  \defaultfontfeatures{Ligatures=TeX,Scale=MatchLowercase}
\fi
% use upquote if available, for straight quotes in verbatim environments
\IfFileExists{upquote.sty}{\usepackage{upquote}}{}
% use microtype if available
\IfFileExists{microtype.sty}{%
\usepackage{microtype}
\UseMicrotypeSet[protrusion]{basicmath} % disable protrusion for tt fonts
}{}
\usepackage[margin=0.5cm]{geometry}
\usepackage{hyperref}
\hypersetup{unicode=true,
            pdftitle={Commandes usuelles en MAXIMA},
            pdfborder={0 0 0},
            breaklinks=true}
\urlstyle{same}  % don't use monospace font for urls
\usepackage{graphicx,grffile}
\makeatletter
\def\maxwidth{\ifdim\Gin@nat@width>\linewidth\linewidth\else\Gin@nat@width\fi}
\def\maxheight{\ifdim\Gin@nat@height>\textheight\textheight\else\Gin@nat@height\fi}
\makeatother
% Scale images if necessary, so that they will not overflow the page
% margins by default, and it is still possible to overwrite the defaults
% using explicit options in \includegraphics[width, height, ...]{}
\setkeys{Gin}{width=\maxwidth,height=\maxheight,keepaspectratio}
\IfFileExists{parskip.sty}{%
\usepackage{parskip}
}{% else
\setlength{\parindent}{0pt}
\setlength{\parskip}{6pt plus 2pt minus 1pt}
}
\setlength{\emergencystretch}{3em}  % prevent overfull lines
\providecommand{\tightlist}{%
  \setlength{\itemsep}{0pt}\setlength{\parskip}{0pt}}
\setcounter{secnumdepth}{0}
% Redefines (sub)paragraphs to behave more like sections
\ifx\paragraph\undefined\else
\let\oldparagraph\paragraph
\renewcommand{\paragraph}[1]{\oldparagraph{#1}\mbox{}}
\fi
\ifx\subparagraph\undefined\else
\let\oldsubparagraph\subparagraph
\renewcommand{\subparagraph}[1]{\oldsubparagraph{#1}\mbox{}}
\fi

%%% Use protect on footnotes to avoid problems with footnotes in titles
\let\rmarkdownfootnote\footnote%
\def\footnote{\protect\rmarkdownfootnote}

%%% Change title format to be more compact
\usepackage{titling}

% Create subtitle command for use in maketitle
\newcommand{\subtitle}[1]{
  \posttitle{
    \begin{center}\large#1\end{center}
    }
}

\setlength{\droptitle}{-2em}
  \title{Commandes usuelles en \texttt{MAXIMA}}
  \pretitle{\vspace{\droptitle}\centering\huge}
  \posttitle{\par}
  \author{}
  \preauthor{}\postauthor{}
  \date{}
  \predate{}\postdate{}

\usepackage{longtable}
\usepackage{booktabs}
\usepackage{multirow}

\begin{document}
\maketitle

\begin{center}
\begin{longtable}{@{}p{4cm}p{10cm}cc@{}}
\toprule
\textbf{Sujet} & \textbf{Discussion} & \textbf{Entrée \texttt{MAXIMA}} & \textbf{Sortie \texttt{MAXIMA}} \\ 
\midrule
\endhead

\textbf{Entrée de commandes} & Vous devez utiliser le point virgule (\texttt{;}) et \texttt{Ctrl-Entrée} pour effectuer une commande &
5/6*5/6; & $\frac{25}{36}$ \\

\midrule

\textbf{Utilisation de la ligne précédente} & Le symbole (\texttt{\%}) dit à \texttt{MAXIMA} d'utiliser le calcul précédent. &
\%+1; & $\frac{61}{36}$ \\

\midrule

\textbf{Utilisation d'une ligne  par son nom} & Vous pouvez utiliser un résultat à l'aide de sa ligne de sortie. &
\%o2+1; & $\frac{97}{36}$ \\

\midrule

\textbf{Évaluation numérique} & Vous pouvez demander à \texttt{MAXIMA} de calculer la valeur numérique d'un résultat. &
\texttt{float(\%)}; & $2.694444444444445$ \\

\midrule

\multirow{4}{*}{\textbf{Caractères spéciaux}} & $\pi$ : \texttt{\%pi} & \texttt{float(\%pi)}; & $3.141592653589793$ \\
& $e$ : \texttt{\%e} & \texttt{float(\%e)}; & $2.718281828459045$ \\
& $\infty$ : \texttt{inf} & \texttt{inf}; & $\infty$ \\
& $-\infty$ : \texttt{minf} & \texttt{minf}; & $-\infty$ \\

\midrule

\multirow{5}{*}{\textbf{Évaluation numérique}} & Addition : \texttt{+} & \texttt{1+3}; & $4$ \\
& Soustraction : \texttt{-} & \texttt{5-10}; & $-5$ \\
& Multiplication : \texttt{+} & \texttt{2*3}; & $6$ \\
& Division : \texttt{/} & \texttt{5/25}; & $\frac{1}{5}$ \\
& Puissance : $\^$ & \texttt{3\^2}; & $9$ \\

\midrule

\textbf{Définir une fonction} & Pour définir une fonction, vous lui donnez un nom, suivi par sa variable indépendante entre parenthèses, suivis par le symbole \texttt{:=}, suivi par sa définition. &
\texttt{f1(x):=x\^2-5*x+6}; & $f1(x):=x^2-5x+6$ \\

\midrule

\textbf{Évaluation d'une fonction} & Une fois définie, vous pouvez utiliser la fonction de la même façon que vous le feriez habituellement. &
\texttt{f1(5)}; & $6$ \\

\midrule

\multirow{2}{4cm}{\textbf{Assignation d'une valeur à une variable}} & Le symbole \texttt{:} assigne une valeur à une variable. & \texttt{a:5}; & $5$ \\
& La valeur associée à la variable sera maintenant utilisée à la place du nom de la variable. & \texttt{f1(a)}; & $6$ \\

\midrule

\textbf{Définir une équation} & Le symbole \texttt{=} défini une équation. & \texttt{x\^2+5*x+6=0}; & $x^2+5x+6=0$ \\

\midrule

\textbf{Assignation d'une expression à une variable} & Vous pouvez utiliser le symbole \texttt{:} pour assigner une équation à une variable. & \texttt{e1:x\^2+5*x+6=0}; & $x^2+5x+6=0$ \\

\midrule

\multirow{3}{4cm}{\textbf{Résoudre une équation}} & Vous pouvez utiliser la commande \texttt{solve} pour résoudre une équation. Vous entrez en premier lieu l'équation à résoudre, suivie par une virgule et la variable par rapport à laquelle vous voulez résoudre. & \texttt{solve(2*x+1=3,x)}; & $[x=1]$ \\
& Un autre exemple qui résout une équation quadratique. & \texttt{sol:solve(x\^2+2*x-3=0,x)}; & $[x=-3,x=1]$ \\
& Vous pouvez utiliser les résultats obtenus à l'aide de la commande \texttt{solve} en effectuant une extraction. & \texttt{sol[2]}; & $x=1$ \\

\midrule

\textbf{Substitution de résultats d'un calcul dans un autre} & Nous pouvons éviter de retaper des résultats en utilisant une substitution \texttt{subst}. Vous entrez en premier lieu la valeur à substituer suivie d'une virgule. Vous entrez enfin l'équation dans laquelle vous voulez substituer la valeur. & \texttt{subst(sol[2],y=3*x-11)}; & $-8$ \\

\midrule

\textbf{Factorisation de polynômes} & Il est possible de factoriser des polynômes avec \texttt{MAXIMA} en utilisant la commande \texttt{factor}. & \texttt{factor(x\^2+2*x+1)}; & $(x+1)^2$ \\

\midrule

\textbf{Développement de polynômes} & Il est possible de développer des polynômes avec \texttt{MAXIMA} en utilisant la commande \texttt{expand}. & \texttt{expand((x+3)*(x+1))}; & $x^2+4x+3$ \\

\midrule

\textbf{Simplification d'expressions trigonométriques} & Il est possible de simplifier des expressions trigonométriques avec \texttt{MAXIMA} en utilisant la commande \texttt{trigsimp}. & \texttt{trigsimp(sin(x)\^2+cos(x)\^2)}; & $1$ \\

\midrule

\textbf{Développement d'expressions trigonométriques} & Il est possible de développer des expressions trigonométriques avec \texttt{MAXIMA} en utilisant la commande \texttt{trigexpand}. & \texttt{trigexpand(cos(x+3*y))}; & $\cos{(x)}\,\cos{(3y)}-\sin{(x)}\,\sin{(3y)}$ \\

\midrule

\textbf{Simplification d'expressions rationnelles} & Il est possible de simplifier des expressions rationnelles avec \texttt{MAXIMA} en utilisant la commande \texttt{fullratsimp}. & \tiny{\texttt{fullratsimp((x\^2+2*x+1)/(x+1)+1/(4*x+3))}}; & $\frac{4{{x}^{2}}+7x+4}{4x+3}$ \\

\midrule

\textbf{Développement d'expressions rationnelles} & Il est possible de développer des expressions rationnelles avec \texttt{MAXIMA} en utilisant la commande \texttt{ratexpand}. & \texttt{ratexpand((x\^2-1)/(x+2))}; & $\frac{{{x}^{2}}}{x+2}-\frac{1}{x+2}$ \\

\bottomrule
\end{longtable}
\end{center}


\end{document}
