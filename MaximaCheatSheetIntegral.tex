\documentclass[8pt,landscape]{article}
\usepackage{lmodern}
\usepackage{amssymb,amsmath}
\usepackage{ifxetex,ifluatex}
\usepackage{fixltx2e} % provides \textsubscript
\ifnum 0\ifxetex 1\fi\ifluatex 1\fi=0 % if pdftex
  \usepackage[T1]{fontenc}
  \usepackage[utf8]{inputenc}
\else % if luatex or xelatex
  \ifxetex
    \usepackage{mathspec}
  \else
    \usepackage{fontspec}
  \fi
  \defaultfontfeatures{Ligatures=TeX,Scale=MatchLowercase}
\fi
% use upquote if available, for straight quotes in verbatim environments
\IfFileExists{upquote.sty}{\usepackage{upquote}}{}
% use microtype if available
\IfFileExists{microtype.sty}{%
\usepackage{microtype}
\UseMicrotypeSet[protrusion]{basicmath} % disable protrusion for tt fonts
}{}
\usepackage[margin=0.5cm]{geometry}
\usepackage{hyperref}
\hypersetup{unicode=true,
            pdftitle={Commandes usuelles pour calcul intégral en MAXIMA},
            pdfborder={0 0 0},
            breaklinks=true}
\urlstyle{same}  % don't use monospace font for urls
\usepackage{graphicx,grffile}
\makeatletter
\def\maxwidth{\ifdim\Gin@nat@width>\linewidth\linewidth\else\Gin@nat@width\fi}
\def\maxheight{\ifdim\Gin@nat@height>\textheight\textheight\else\Gin@nat@height\fi}
\makeatother
% Scale images if necessary, so that they will not overflow the page
% margins by default, and it is still possible to overwrite the defaults
% using explicit options in \includegraphics[width, height, ...]{}
\setkeys{Gin}{width=\maxwidth,height=\maxheight,keepaspectratio}
\IfFileExists{parskip.sty}{%
\usepackage{parskip}
}{% else
\setlength{\parindent}{0pt}
\setlength{\parskip}{6pt plus 2pt minus 1pt}
}
\setlength{\emergencystretch}{3em}  % prevent overfull lines
\providecommand{\tightlist}{%
  \setlength{\itemsep}{0pt}\setlength{\parskip}{0pt}}
\setcounter{secnumdepth}{0}
% Redefines (sub)paragraphs to behave more like sections
\ifx\paragraph\undefined\else
\let\oldparagraph\paragraph
\renewcommand{\paragraph}[1]{\oldparagraph{#1}\mbox{}}
\fi
\ifx\subparagraph\undefined\else
\let\oldsubparagraph\subparagraph
\renewcommand{\subparagraph}[1]{\oldsubparagraph{#1}\mbox{}}
\fi

%%% Use protect on footnotes to avoid problems with footnotes in titles
\let\rmarkdownfootnote\footnote%
\def\footnote{\protect\rmarkdownfootnote}

%%% Change title format to be more compact
\usepackage{titling}

% Create subtitle command for use in maketitle
\newcommand{\subtitle}[1]{
  \posttitle{
    \begin{center}\large#1\end{center}
    }
}

\setlength{\droptitle}{-2em}
  \title{Commandes usuelles pour calcul intégral en \texttt{MAXIMA}}
  \pretitle{\vspace{\droptitle}\centering\huge}
  \posttitle{\par}
  \author{}
  \preauthor{}\postauthor{}
  \date{}
  \predate{}\postdate{}

\usepackage{longtable}
\usepackage{booktabs}
\usepackage{multirow}

\begin{document}
\maketitle

\begin{center}
\begin{longtable}{p{4cm}p{10cm}cc}
\toprule
\textbf{Sujet} & \textbf{Discussion} & \textbf{Entrée \texttt{MAXIMA}} & \textbf{Sortie \texttt{MAXIMA}} \\ 
\midrule
\endhead

\multirow[c]{3}{4cm}{\textbf{Calculer une limite}} & Vous pouvez utiliser la commande \texttt{limit} pour calculer une limite. Vous entrez en premier lieu la fonction dont on cherche la limite, la variable et enfin la valeur où nous évaluons la limite. & \texttt{limit(sin(x)/x,x,0)}; & $1$ \\
\cmidrule{2-4}
& La limite à droite est obtenue en ajoutant \texttt{plus}. & \small{\texttt{limit(tan(x),x,\%pi/2,plus)}}; & $-\infty$ \\
\cmidrule{2-4}
& La limite à gauche est obtenue en ajoutant \texttt{minus}. & \small{\texttt{limit(tan(x),x,\%pi/2,minus)}}; & $\infty$ \\

\midrule

\textbf{Dérivation d'une fonction} & Il est possible de dériver des fonctions avec \texttt{MAXIMA} en utilisant la commande \texttt{diff}. Nous devons spécifier la fonction et la variable indépendante. & \texttt{diff((3*x+2)$\wedge$3,x)}; & $9{{\left( 3x+2\right) }^{2}}$ \\

\midrule

\textbf{Dérivation d'ordre supérieur d'une fonction} & Il est possible de trouver les dérivées d'ordres supérieurs de fonctions avec \texttt{MAXIMA} en utilisant la commande \texttt{diff} et en ajoutant l'ordre de la dérivée. & \texttt{diff((3*x+2)$\wedge$3,x,3)}; & $54\left( 3x+2\right)$ \\

\midrule

\multirow{2}{4cm}{\textbf{Dérivations implicites}} & Nous pouvons dériver implicitement avec \texttt{MAXIMA} en utilisant la commande \texttt{diff}. Nous spécifions l'équation à dériver en utilisant la commande \texttt{diff} et en indiquant quelle variable dépendante dépend de la variable indépendante. & \texttt{deriv:diff(x$\wedge$2+y(x)$\wedge$2,x)}; & $2\operatorname{y}(x)\,\left( \frac{d}{dx}\operatorname{y}(x)\right) +2x$ \\
\cmidrule{2-4}
& Nous isolons la dérivée en utilisant la commande \texttt{solve}. & \texttt{solve(deriv,'diff(y(x),x))}; & $[\frac{d}{dx}\operatorname{y}(x)=-\frac{x}{\operatorname{y}(x)}]$ \\

\midrule

\textbf{Intégrale indéfinie} & Il est possible de trouver des intégrales indéfinies avec \texttt{MAXIMA} en utilisant la commande \texttt{integrate}. Nous devons spécifier la fonction et la variable indépendante. & \texttt{integrate(cos(3*x),x)}; & $\frac{\sin(3x)}{3}$ \\

\midrule

\textbf{Intégrale définie} & Il est possible de trouver des intégrales définies avec \texttt{MAXIMA} en utilisant la commande \texttt{integrate}. Nous devons spécifier la fonction, la variable indépendante et les deux bornes. & \texttt{integrate(cos(3*x),x,0,\%pi/2)}; & $-\frac{1}{3}$ \\

\midrule

\textbf{Intégrale numérique} & Il est possible de trouver des intégrales numérique avec \texttt{MAXIMA} en utilisant la commande \texttt{romberg}. Nous devons spécifier la fonction, la variable indépendante et les deux bornes. & \texttt{romberg(sin(sin(x)),x,0,1)}; & $0.4306059236425572$ \\

\midrule

\multirow{2}{4cm}{\textbf{Équations différentielles du premier ordre}} & Nous pouvons résoudre des équations différentielles du premier ordre avec \texttt{MAXIMA} en utilisant la commande \texttt{ode2}. Nous spécifions l'équation différentielle en utilisant la commande \texttt{diff} et en la faisant précéder d'une apostrophe \texttt{'}. & \texttt{edo1:ode2('diff(y,x)=3*y, y, x)}; & $y=\mathit{\%{}c}\,{{\%{}e}^{3x}}$ \\
\cmidrule{2-4}
& Lorsque nous avons une condition initiale nous utilisons la commande \texttt{ic1}. & \texttt{ic1(edo1, x=0, y=2)}; & $y=2\%e^{3x}$ \\

\midrule

\multirow{2}{4cm}{\textbf{Équations différentielles du second ordre}} & Nous pouvons résoudre des équations différentielles du second ordre avec \texttt{MAXIMA} en utilisant la commande \texttt{ode2}. Nous spécifions l'équation différentielle en utilisant la commande \texttt{diff} et en la faisant précéder d'une apostrophe \texttt{'}. & \texttt{edo2:ode2('diff(y,x,2)=y, y, x)}; & $y=\mathit{\%{}k1}\,{{\%{}e}^{x}}+\mathit{\%{}k^2}\,{{\%{}e}^{-x}}$ \\
\cmidrule{2-4}
& Lorsque nous avons une condition initiale nous utilisons la commande \texttt{ic2}. & \texttt{ic2(edo2,x=0,y=2,'diff(y,x)=1)}; & $y=\frac{3{{\%{}e}^{x}}}{2}+\frac{{{\%{}e}^{-x}}}{2}$ \\

\midrule

\multirow{2}{4cm}{\textbf{Sommations}} & Nous pouvons écrire la sommation avec \texttt{MAXIMA} en utilisant la commande \texttt{sum}. Nous spécifions le terme général, la variable, le terme initial et le terme final. & \texttt{sum(1/k$\wedge$2,k,1,inf)}; & $\sum\limits_{k=1}^{\infty} \frac{1}{k^2}$ \\
\cmidrule{2-4}
& Pour évaluer la sommation nous utilisons la commande \texttt{simpsum}. & \texttt{sum(1/k$\wedge$2,k,1,inf)}, \texttt{simpsum}; & $\frac{\pi^2}{6}$ \\

\midrule

\textbf{Séries de Taylor} & Nous pouvons la série de Taylor d'une fonction avec \texttt{MAXIMA} en utilisant la commande \texttt{taylor}. Nous spécifions la fonction, la variable, autour de quelle valeur nous calculons notre série et la puissance maximale du polynôme voulu. & \texttt{taylor(sin(x),x,0,5)}; & $x-\frac{{{x}^{3}}}{6}+\frac{{{x}^{5}}}{120}+\mbox{...}$ \\

\bottomrule
\end{longtable}
\end{center}


\end{document}
